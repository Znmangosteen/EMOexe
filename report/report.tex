\documentclass[conference]{IEEEtran}

  \usepackage{booktabs}
  \usepackage{listing}
  \usepackage{amsmath}
  \usepackage{algorithm}
  \usepackage{array}
  \usepackage{url}
  \usepackage{cite}
  \usepackage{complexity}
  \usepackage{algpseudocode}
  \usepackage{graphicx}
% \usepackage{algorithm}
  \ifCLASSINFOpdf
  
  \else
  
  \fi
  
  \hyphenation{op-tical net-works semi-conduc-tor}
  
  
  \begin{document}
  
  \title{Parallel Distributed Multi-objective Fuzzy Genetics-based Machine Learning \\ Mid Term Report}
  
  \author{\IEEEauthorblockN{Bowen Zheng, Shijie Chen, Shuxin Wang}
  \IEEEauthorblockA{Department of Computer Science and Engineering\\
  Southern University of Science and Technology\\
  Shenzhen, Guangdong, China\\}
  }
  
  \maketitle
  
  \begin{abstract}
  In the second period of the project, we finished the design of fuzzy classifiers, GBML framework and the asynchronous parallel distributed system. We have implemented each part respectively and are working to integrate them together.
  \end{abstract}
  \IEEEpeerreviewmaketitle
  
  \section{Introduction}
  In this project, we aim to build a parallel distributed implementation of a multi-objective genetics based machine learning(GBML) algorithm. We choose a specific problem of a three-objective fuzzy rule-based classifier and fit it into a hybrid GBML framework. Then we develop a parallel mechanism to accelerate computation.

  Code of the three parts have been completed. We will integrate them together, fix bugs and run some test problems in the next stage.

  \section{Fuzzy Rule-based Classifiers}
  The design and implementation of fuzzy classifier is based on \cite{ishibuchi2007analysis}.
  \subsection{Fuzzy Rules}
  We use the following "if-then" rules:
  $$Rule\;R_q:\;if\;x_{pi}\;is\;A_{qi},\;i\in [1,n]\;then\;Class\;C_q\;with\;CF_q$$

  \begin{table}[H]
    \caption{Notation}
    \centering
      \begin{tabular}{cccc}
      \toprule
      Variable&Name\\
      \midrule
      $n$&dimension of patterns\\
      $M$&number of classes of patterns\\
      $S$&a fuzzy classifier\\
      $I$&number of membership functions\\
      $x_p$&a pattern vector\\
      $x_{pi}$&attribute of $x_p$ value on $i$-th dimension\\
      $A_{qi}$&antecedent fuzzy set\\
      $\mu_{A_{qi}}(x)$&membership function of $A_{qi}$\\
      $\mu_{A_{q}}(x_p)$&compatibility grade of $x_p$ with $A_q$\\
      $A_q$&antecedent part of $q$-th rule\\
      
      $R_q$&$q$-th rule\\
      $C_q$&consequent class for $q$-th rule\\
      $CF_q$&rule weight for $q$-th rule\\
  \bottomrule
  \end{tabular}
  \label{table:1}
  \end{table}
  Input vectors are normalized to a hypercube $[0,1]^n$ using the following equation:
  $$x_{pi}=\frac{x_{pi}-min(x_{i})}{max(x_{i})-min(x_{i})}$$

  The antecedent fuzzy set contains a membership function of the form:
  $$\mu_{A_{qi}}(x_{pi}) =max\{1-\frac{|a-x_{pi}|}
  {b},0\}$$
  $$a=\frac{k-1}{K-1}$$
  $$b=\frac{1}{K-1}$$
  Where K is the number of intervals of fuzzy set $A_{qi}$ and k is the order of the interval. As shown in fig.\ref{fig:mfunc}. The $don't\;care$ condition is a constant function with value 1. As we will discuss in classification process, the feature $x_{pi} $ with $A_{qi}$ being $don't\;care$ is ignored.

  \begin{figure}[H]
    \centering
    \includegraphics[width = 0.4\textwidth]{figures/mfunc.png}
    \caption{Membership functions with at most 3 intervals}
    \label{fig:mfunc}
  \end{figure}

  The antecedent part of $R_q$ is a set of antecedent fuzzy sets. $$A_q = \{A_{qi}|i\in [1,n]\}$$

  We define the compatibility grade of an pattern $x_p$ with rule $R_q$ as
  $$\mu_{A_q}(x_p) = \prod_{i = 1}^{n}\mu_{A_{qi}}(x_{pi})$$

  Then we determine the consequent class $C_q$ and rule weight $CF_q$ using training patterns as follows:

  First, we compute the confidence of fuzzy rule $R_q$ to each class $c(A_q \Rightarrow Class\;h), h \in[1,M]$
  $$c(A_q \Rightarrow Class\; h) = \frac{\sum\limits_{x_p \in Class\;h}\mu_{A_q}(x_p)}{\sum\limits_{p=1}^m\mu_{A_q}(x_p)}$$

  Then the consequent class $C_q$ as the class with which $R_q$ has the largest confidence.
  $$c(A_q \Rightarrow Class\;C_q)=max\{c(A_q \Rightarrow Class\;h)|h \in [1,M]\}$$
  
  At last, rule weight of $R_q$ is given by the difference between its consequent class and other classes.
  $$CF_q = c(A_q\Rightarrow Class\;C_q)-\sum\limits_{h=1, h\neq C_q}^{M}c(A_q\Rightarrow Class \; h)$$

  Rule weight shows the quality of a classification result given by a fuzzy rule. If a rule has negative rule weight, its abandoned.
  
  \subsection{Fuzzy Classifier}
  
  A fuzzy classifier $S$ is a set of fuzzy rules. Given an input pattern $x_p$, the classification result $C_w$ is produced by a winning rule $R_w$, chosen as follows:
  $$\mu_{A_w}(x_p) \cdot CF_w = max\{\mu_{A_q}(x_p)\cdot CF_q|R_q\in S\}$$
  \subsection{Generate Fuzzy Rule From Training Patterns}
  At the initiation stage of our GBML algorithm, the population, set of fuzzy classifiers, is created from training data. We can use certain training patterns to create a fuzzy classifier and repeat the process to create a set of classifiers.
  
  Each rule $R_t$ in each classifier is generated from a training pattern $x_t$ as follows:
% \begin{itemize}
    % \item

  We choose an antecedent fuzzy set according to each $x_{ti}$ and form the antecedent part $A_t$ of rule $R_t$ so that $x_{t}$ has the largest compatibility:
  $$ \A_{ti} = \underset{\mu_j}{\arg\max}\{\mu_{j}(x_{ti})\},j \in [1,I]$$
    % \item 
  
  Then, randomly change the antecedent fuzzy set to $dont't\;care$ according to a pre-specified probability $p_{dc}$. This step prevents overfitting and improves generalizing ability of fuzzy rules.
% \end{itemize}
  \section{Hybrid Genetics-based Machine Learning Framework}
	 
  \section{Asynchronous Parallel Distributed System Design}


  \section{Contribution}
    \begin{itemize}
    \item Bowen Zheng - Design \& Implementation of parallel system
    \item Shijie Chen - Design \& Implementation of fuzzy classifier, Design of parallel system
    \item Shuxin Wang - Design \& Implementation of Hybrid GBML framework
    \end{itemize}
    

  \section*{Acknowledgment}


\bibliographystyle{ieeetr}
\bibliography{ref}
% that's all folks
\end{document}


  